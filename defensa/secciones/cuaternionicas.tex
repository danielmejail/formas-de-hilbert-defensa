\theoremstyle{remark}
\newtheorem{teoNorma}{Teorema}[section]
\newtheorem{defAlgebraDeCuaterniones}[teoNorma]{Definici\'{o}n}
\newtheorem{ejemMatrices}[teoNorma]{Ejemplos}
\newtheorem{teoClasificacion}[teoNorma]{Teorema}
\newtheorem{obsExtension}[teoNorma]{Observaci\'{o}n}
\newtheorem{defRamifica}[teoNorma]{Definici\'{o}n}
\newtheorem{defOrdenesEIdeales}[teoNorma]{Definiciones}
\newtheorem{lemaDiscriminanteReducido}[teoNorma]{Lema}
\newtheorem{obsDiscriminanteReducido}[teoNorma]{Observaci\'{o}n}
\newtheorem{defOrdenDeEichler}[teoNorma]{Definici\'{o}n}
\newtheorem{defCuaternionicas}[teoNorma]{Definici\'{o}n}
\newtheorem{obsCuaternionicas}[teoNorma]{Observaciones}
\newtheorem{teoCorrespondenciaJL}[teoNorma]{Teorema}
\newtheorem{obsCorrespondenciaJL}[teoNorma]{Observaci\'{o}n}

%-------------

\subsection{\'{A}lgebras de cuaterniones}\label{subsec:preliminares}

\begin{frame}{\'{A}lgebras de cuaterniones}
	\begin{defAlgebraDeCuaterniones}\label{def:algebradecuaterniones}
		Un \'{a}lgebra de cuaterniones sobre un cuerpo $F$
		($\car(F)\not=2$) es una $F$-\'{a}lgebra $B$ con dos
		generadores $i,j$ que verifican
		\begin{align*}
			i^2\,=\,a & \text{ ,}\quad j^2\,=\,b
				\quad\text{y}\quad ji\,=\,-ij
		\end{align*}
		%
		$a,b\in F^\times$, $k:=ij$, $B=\quatalg[F]{a,b}$. Si $h\in B$,
		es ra\'{\i}z de
		\begin{align*}
			X^2\,-\,\trd(h)\,X\,+\,\nrd(h) & \,=\,
				X^2\,-\,(h+\conj h)\,X\,+\,\conj hh
			\text{ .}
		\end{align*}
		%
	\end{defAlgebraDeCuaterniones}
	\begin{ejemMatrices}\label{ejem:matrices}
		\begin{itemize}
			\item $\MM[2\times 2](\bb Q)$:
				\begin{math}
					i=\left[\begin{smallmatrix}
						& 1 \\ 1 &
					\end{smallmatrix}\right]
				\end{math},
				\begin{math}
					j=\left[\begin{smallmatrix}
						-1 & \\ & 1
					\end{smallmatrix}\right]
				\end{math}, $\nrd=\det$ y
				$\trd=\traza$;
			\item $\quatalg[\bb Q]{-1,-1}\subset\bb H$.
		\end{itemize}
	\end{ejemMatrices}
\end{frame}

\begin{frame}{\'{A}lgebras de cuaterniones (cont.)}
	\begin{obsExtension}\label{obs:extension}
		Dadas $B/F$ y una extensi\'{o}n $K/F$, $B\otimes_F K$ es un
		\'{a}lgebra de cuaterniones sobre $K$. (%
		\begin{math}
			\quatalg[\bb Q]{-1,-1}\otimes_{\bb Q}\bb R\simeq\bb H
		\end{math}).
		%
	\end{obsExtension}

	\begin{defRamifica}\label{def:ramifica}
		\begin{itemize}
			\item\emph{$B$ ramifica en $v$}, si $B_v/\bb Q_v$ es de
				divisi\'{o}n ($v=p,\infty$);
			\item\emph{discriminante}:
				\begin{math}
					\disc[B]:=\prod_{p\in\Ram(B)}\,
						\generado{p}
				\end{math}. (%
				\begin{math}
					\disc[{\quatalg[\bb Q]{-1,-1}}]=2\bb Z
				\end{math});
			\item $B$ es \emph{indefinida}, si
				$B_\infty\simeq\MM[2\times2](\bb R)$;
			\item $B$ es \emph{definida}, si
				$B_{\infty}\simeq\bb H$.
		\end{itemize}
	\end{defRamifica}

	\begin{teoClasificacion}[Clasifici\'{o}n global]%
		\label{thm:clasificacionglobal}
		Sobre un cuerpo de n\'{u}meros,
		\begin{math}
			\big[B/F\big] \,\leftrightarrow\,
				\text{subconjuntos pares de }\lugares{F}
		\end{math}
	\end{teoClasificacion}
\end{frame}

\begin{frame}{Ejemplos}
	\begin{center}
		\begin{tabular}{c|ccc}
			$B$ & $\Ram(B)$ & $\disc[B]$ & $B_\infty$ \\
			\hline
			$\quatalg[\bb Q]{-1,-11}$ & $\{11,\infty\}$ &
				$11\cdot\bb Z$ & $\bb H$ \\
			$\quatalg[\bb Q(\sqrt 5)]{-1,-1}$ &
				$\{\tau_1,\tau_2\}$ & $1$ &
				$B_{\tau_1},B_{\tau_2}\simeq\bb H$ \\
			$\quatalg[\bb Q(w)]{5- w^2,1-w^2}$ &
				$\{\tau_1,\tau_3\}$ & $1$ &
				$B_{\tau_2}\simeq\MM[2\times 2](\bb R)$
		\end{tabular}
	\end{center}
	\begin{align*}
		& w^3 - w^2 - 8\,w + 7 \,=\,0 \\
		& \quad\tau_1 \,:\,w\,\mapsto\,-2,781\dots \text{ ,} \\
		& \quad\tau_2 \,:\,w\,\mapsto\,0,8621\dots \text{ ,} \\
		& \quad\tau_3 \,:\,w\,\mapsto\,2,919\dots
	\end{align*}
	%
\end{frame}

\begin{frame}{\'{O}rdenes e ideales}
	\begin{defOrdenesEIdeales}\label{def:ordeneseideales}
		\begin{itemize}
			\item Un \emph{ret\'{\i}culo (completo)} en $B$ es un
				$\bb Z$-m\'{o}dulo $I\subset B$ f.g. que
				contiene una base ($\bb Q\cdot I=B$);
			\item $x\in B$ es \emph{\'{\i}ntegro}, si
				$\nrd(x),\trd(x)\in\bb Z$;
			\item un \emph{orden} es un ret\'{\i}culo $\cal O$ que
				es subanillo (con $1$);
			\item
				\begin{math}
					\Oder(I):=\big\{h\in B\,:\,
						I\,h\subset I\big\}
				\end{math} es un orden;
			\item un \emph{$\cal O$-ideal (a derecha)} es un
				ret\'{\i}culo $I$ tal que $\Oder(I)=\cal O$.
		\end{itemize}
		%
	\end{defOrdenesEIdeales}
\end{frame}

% \begin{frame}{\'{O}rdenes e ideales (cont.)}
	% % Todo orden tiene asociado un ideal que mide, entre otras cosas,
	% % cu\'{a}n lejos est\'{a} el orden de ser maximal.
	% Dados $\beta_1,\,\beta_2,\,\beta_3,\,\beta_4\in B$,
	% \begin{align*}
		% \disc(\beta_1,\beta_2,\,\beta_3,\,\beta_4) & \,:=\,
			% \det\,\trd(\beta_i\beta_j)
		% \text{ .}
	% \end{align*}
	% %
	% El \emph{discriminante (reducido)} de $\cal O$ es el ideal
	% \begin{align*}
		% \drd(\cal O) & \,:=\,\sqrt{\generado{\disc(%
			% \beta_1,\,\beta_2,\,\beta_3,\,\beta_4)\,:\,%
			% \beta_i\in \cal O}}
		% \text{ .}
	% \end{align*}
	% %
	% % Esto requiere algunas explicaciones: si $\{\beta_i\}_i$ es un
	% % conjunto generador de $\cal O$ (pseudobase), entonces
	% % $\disc(\alpha_i)=\det(M)^2\,\disc(\beta_i)$, $M$ la matriz de
	% % coeficientes de los $\alpha_i$ en t\'{e}rminos de los $\beta_i$.
	% % En general, $\det(M)\in\frak a$, para cierto ideal (?`\'{\i}ntegro?)
	% % del cuerpo de base y el discriminante (no reducido) de $\cal O$
	% % es $\frak a^2\,\disc(\beta_i)$ (o algo del estilo). Adem\'{a}s,
	% % se puede ver (usando la relaci\'{o}n con el \'{\i}ndice y que existe
	% % un orden con discriminante cuadrado) que el discriminante de
	% % cualquier orden (ret\'{\i}culo) es un cuadrado. El discriminante
	% % reucido es igual al ideal que se obtiene dividiendo los exponentes.
	% %
	% \begin{lemaDiscriminanteReducido}\label{lema:discriminantereducido}
		% Si $\cal O\subset\cal O'$, entonces
		% \begin{math}
			% \drd(\cal O)\subset\drd(\cal O')
		% \end{math}.
		% En ese caso, $\cal O=\cal O'$, si y s\'{o}lo si
		% $\drd(\cal O)=\drd(\cal O')$.
	% \end{lemaDiscriminanteReducido}
% \end{frame}

\begin{frame}{\'{O}rdenes e ideales (cont.)}
	\begin{lemaDiscriminanteReducido}\label{lema:discriminantereducido}
		Todo orden $\cal O\subset B$ tiene asociado un ideal
		$\drd(\cal O)\subset F$ de manera que, si
		$\cal O\subset\cal O'$, entonces
		\begin{math}
			\drd(\cal O)\subset\drd(\cal O')
		\end{math}.
		En ese caso, $\cal O=\cal O'$, si y s\'{o}lo si
		$\drd(\cal O)=\drd(\cal O')$.
	\end{lemaDiscriminanteReducido}
	\begin{obsDiscriminanteReducido}\label{obs:discriminantereducido}
		Existen \'{o}rdenes maximales.
	\end{obsDiscriminanteReducido}
	\begin{defOrdenDeEichler}\label{def:ordendeeichler}
		Un \emph{orden de Eichler} es la intersecci\'{o}n de dos
		\'{o}rdenes maximales. Si $\cal O$ es de Eichler,
		\begin{align*}
			\drd(\cal O) & \,=\,\disc[B]\cdot\frak N
			\text{ ,}
		\end{align*}
		%
		$\frak N$ es el \emph{nivel} del orden y
		$(\disc[B],\frak N)=1$.
	\end{defOrdenDeEichler}
\end{frame}

\begin{frame}{Ejemplos}
	\begin{itemize}
		\item $B=\MM[2\times 2](F)$,
			\begin{align*}
				\cal O\,:=\,
				\left[\begin{matrix}
					\oka F & \oka F \\ \frak N & \oka F
				\end{matrix}\right] \,\subset\,
				\MM[2\times 2](\oka F)
				& \qquad \text{(\phantom)}
					\cal O_+^\times=\Gamma_0(\frak N)
					\text{\phantom();}
			\end{align*}
			%
		\item $B=\quatalg[\bb Q]{-1,-11}$, $\generado{1,i,j,k}$
			no es maximal:
			\begin{center}
				\begin{tabular}{c|c}
					$\cal O$ &
						\begin{math}
							\disc(\beta_1,\,
								\beta_2,\,
								\beta_3,\,
								\beta_4)
						\end{math} \\
					\hline
					$\generado{1,i,j,k}$ &
						$4^2\,11^2$ \\
					$\generado{1,i,j,\frac{1+i+j+k}{2}}$ &
						$2^2\,11^2$ \\
					$\generado{1,i,\frac{1+j}{2},%
						\frac{i+k}{2}}$ & $11^2$ \\
					$\generado{1,i,\frac{1+k}{2},%
						\frac{i-j}{2}}$ & $11^2$
				\end{tabular}
			\end{center}
	\end{itemize}
\end{frame}

\begin{frame}{Clases de ideales}
	\begin{align*}
		\ideales{\cal O} & \,:=\,\big\{I\subset B\,:\,
			\Oder(I)=\cal O	\text{ , invertible}\big\}
		\text{ .}
	\end{align*}
	%
	Dados $I,J\in\ideales{\cal O}$, $I\sim J$, si existe $h\in B^\times$
	tal que $I=h\,J$.
	\begin{align*}
		\Class{\cal O} & \,:=\,B^\times\backslash\ideales{\cal O}
		\text{ .}
	\end{align*}
	%
	En general, no es un grupo. $H(\cal O):=\#\Class{\cal O}<\infty$.
	\begin{teoNorma}\label{thm:norma}
		Si $\cal O$ es de Eichler,
		\begin{math}
			B_+^\times :=\big\{h\in B^\times\,:\,\nrd(h)\gg 0\big\}
		\end{math},
		\begin{align*}
			\nrd & \,:\,B^\times_+\backslash\ideales{\cal O}
				\,\twoheadrightarrow\,\pClass{F}
			\text{ .}
		\end{align*}
		%
		Si $B$ es indefinida, es una biyecci\'{o}n (aproximaci\'{o}n
		fuerte). Si $B$ es definida,
		$\nrd:\,\Class{\cal O}\twoheadrightarrow\pClass F$.
	\end{teoNorma}
\end{frame}

\subsection{Formas cuaterni\'{o}nicas}\label{subsec:cuaternionicas}

\begin{frame}{Formas cuaterni\'{o}nicas: $B$ indefinida}
	Sean $B/F$ un \'{a}lgebra indefinida ($r=1$) y
	$v_1,\,\dots,\,v_n\in\lugares[\infty]F$:
	\begin{align*}
		B_{v_1}\,\simeq\,\MM[2\times 2](\bb R) & \quad\text{y}\quad
			B_{v_j}\,\simeq\,\bb H
			\quad\text{(\phantom)}j>1\text{\phantom().}
	\end{align*}
	%
	El grupo $B^\times_+=\big\{\nrd\gg 0\big\}$ act\'{u}a en $\hP$ v\'{\i}a
	$\gamma\mapsto\gamma_\infty\in\GLtp[2](\bb R)$. Dadas
	$f:\,\hP\rightarrow\bb C$,
	\begin{math}
		\gamma_\infty=
			\left[\begin{smallmatrix}
				a & b \\ c & d
			\end{smallmatrix}\right]
	\end{math},
	\begin{align*}
		f\operadormatrices{\null}{\gamma}(z) & \,=\,
			\frac{\det(\gamma_\infty)}{(cz+d)^2}\,
				f(\gamma_\infty\cdot z)
		\text{ .}
	\end{align*}
	%
\end{frame}

\begin{frame}{Formas cuaterni\'{o}nicas: $B$ indefinida (cont.)}
	Dado un orden de Eichler $\cal O\subset B$ de nivel $\frak N$,
	$\Gamma:=\cal O_+^\times$. El cociente
	\begin{align*}
		X^B_0(\frak N) & \,:=\,\Gamma\backslash\hP
	\end{align*}
	%
	es una curva compleja compacta (no hay c\'{u}spides) y conexa.

	\begin{defCuaternionicas}\label{def:cuaternionicasindef}
		$f:\,\hP\rightarrow\bb C$ es una \emph{forma cuaterni\'{o}%
		nica de peso $\peso 2$ para $\cal O$}, si
		\begin{enumerate}
			\item $f$ es holomorfa en $\hP$;
			\item para $\gamma\in\Gamma$,
				$f\operadormatrices{}{\gamma}=f$.
		\end{enumerate}
		%
		Denotamos este espacio por $\spitzH[B]2{\frak N}$.
	\end{defCuaternionicas}
\end{frame}

\begin{frame}{Operadores de Hecke}
	El espacio $\spitzH[B]2{\frak N}$ posee un producto interno y, para
	cada primo $\frak p\nmid\disc[B]\cdot\frak N$, operadores
	$T_{\frak p}:\,\spitzH[B]2{\frak N}\rightarrow\spitzH[B]2{\frak N}$
	tales que
	\begin{itemize}
		\item $T_{\frak p}T_{\frak q}=T_{\frak q}T_{\frak p}$ y
		\item
			\begin{math}
				\langle T_{\frak p}f,g\rangle=
					\langle f,T_{\frak p}g\rangle
			\end{math}.
	\end{itemize}
	%
	Si $\frak p=\generado p$ con $p\gg 0$, entonces
	\begin{align*}
		T_{\frak p}f & \,:=\,\sum_{i\in \oka F/\frak p}\,
			f\operadormatrices{}{\pi_i} \,+\,
			f\operadormatrices{}{\pi_\infty}
		\text{ ,}
	\end{align*}
	%
	donde $\pi_i,\pi_\infty\in\cal O$, $\nrd(\pi_i)=\nrd(\pi_\infty)=p$ y
	forman un sistema de representantes de
	\begin{align*}
		\Theta(\frak p) & \,:=\,\Gamma\backslash\Big\{
			\pi\in\cal O_+\,:\,\generado{\nrd(\pi)}=\frak p\Big\}
		\text{ .}
	\end{align*}
	%
\end{frame}

\begin{frame}{Formas cuaterni\'{o}nicas: $B$ definida}
	Sea $B/F$ un \'{a}lgebra definida ($r=0$). $B_{v_i}\simeq\bb H$, si
	$v_i\in\lugares[\infty]{F}$ y no hay acci\'{o}n en $\hP$.
	Sea $\cal O\subset B$ un orden de Eichler de nivel $\frak N$.

	\begin{defCuaternionicas}\label{def:cuaternionicasdef}
		Una \emph{forma modular cuaterni\'{o}nica de peso $\peso 2$ %
		para $\cal O$} es una funci\'{o}n
		$f:\,\ideales{\cal O}\rightarrow\bb C$ tal que $f(b\,I)=f(I)$
		para todo $b\in B^\times$. Denotamos este espacio
		$\modularH[B]2{\frak N}$.
	\end{defCuaternionicas}

	\begin{obsCuaternionicas}\label{obs:cuaternionicasdef}
		Sean $I\in\ideales{\cal O}$, $[I]$ la funci\'{o}n
		carcter\'{\i}stica de la clase.
		\begin{itemize}
			\item $[I]\in\modularH[B]2{\frak N}$;
			\item $\{[I_1],\,\dots,\,[I_H]\}$ es base de
				$\modularH[B]2{\frak N}$.
		\end{itemize}
		%
	\end{obsCuaternionicas}
\end{frame}

\begin{frame}{Operadores de Hecke}
	Dados $\frak p\nmid\disc[B]\cdot\frak N$ e $I\in\ideales{\cal O}$,
	\begin{align*}
		\cal T_{\frak p}(I) & \,:=\,
			\Big\{J\in\ideales{\cal O}\,:\,J\subset I,\,
				\nrd(J)=\frak p\,\nrd(I)\Big\}
			\text{ ,} \\
		T_{\frak p}[I] & \,:=\,\sum_{J\in\cal T_{\frak p}(I)}\,
			[J]
		\text{ .}
	\end{align*}
	%
	\begin{itemize}
		\item El espacio $\modularH2{\frak N}$ admite un producto
			interno respecto del cual los $T_{\frak p}$ son
			autoadjuntos.
		\item Existe $e_0\in\modularH2{\frak N}$, autofunci\'{o}n
			simult\'{a}nea para los $T_{\frak p}$, con autovalor
			$\idnorm\frak p+1$.
		\item
			\begin{math}
				\spitzH[B]2{\frak N}:=
					\Big\{f\in\modularH[B]2{\frak N}\,:\,
					\langle f,e_0\rangle=0\Big\}
			\end{math}.
	\end{itemize}
\end{frame}

\begin{frame}{La correspondencia de Jacquet-Langlands}
	\begin{teoCorrespondenciaJL}\label{thm:correspondenciajl}
		Sea $B/F$ un \'{a}lgebra de cuaterniones de discriminante
		$\disc[B]$ y sea $\frak N'\subset\oka F$ un ideal coprimo con
		$\disc[B]$. Entonces existe un morfismo inyectivo de
		m\'{o}dulos de Hecke
		\begin{align*}
			& \spitzH[B]2{\frak N'} \,\hookrightarrow\,
				\spitzH2{\disc[B]\cdot\frak N'}
		\end{align*}
		%
		cuya imagen consiste en las formas $f$ nuevas en los primos
		$\frak p\mid\disc[B]$.
	\end{teoCorrespondenciaJL}
\end{frame}

\begin{frame}{La correspondencia de Jacquet-Langlands (cont.)}
	Sean $\frak N=\frak p\frak q$ ($\frak p\not =\frak q$), $B/F$ con
	$\Ram(B)\cap\lugares[f]{F}=\{\frak p\}$. Por J-L,
	\begin{align*}
		\spitzH2{\frak p\frak q} & \,=\,
			\spitzH2{\frak p\frak q}^{\frak p-\neue}\,\oplus\,
			\spitzH2{\frak p\frak q}^{\frak p-\oude} \\
		& \,=\,\spitzH[B]2{\frak q}\,\oplus\,
			\Big(\iota_1\big(\spitzH2{\frak q}\big)\,+\,
				\iota_{\frak p}\big(\spitzH2{\frak q}\big)
				\Big)
		\text{ .}
	\end{align*}
	%
	Si $n=[F:\bb Q]=1\text{ o }2$, hay una \'{u}nica posibilidad.
	Si $n>2$ hay muchas \'{a}lgebras (ramificaci\'{o}n en $\infty$).

	Si, en cambio, $\frak N=\frak p^2$, no tenemos tantas opciones: debe
	ser $\disc[B]=1$ y $\Ram(B)\subset\lugares[\infty]F$. Si $n=1$ es
	imposible; si $n=2$, hay una \'{u}nica elecci\'{o}n.
\end{frame}

\begin{frame}{La correspondencia de Jacquet-Langlands (cont.)}
	\begin{obsCorrespondenciaJL}\label{obs:correspondenciajl}
		\begin{itemize}
			\item Si $B/F$ es indefinida,
				$f\in\spitzH[B]2{\frak N}$ tiene asociada una
				$r$-forma diferencial holomorfa en la
				\emph{variedad}
				$\cal O^\times_+\backslash\hP^r$;
			\item si $B$ es totalmente definida, $f$ es una
				funci\'{o}n en un \emph{espacio finito}.
		\end{itemize}
		%
	\end{obsCorrespondenciaJL}
	Hacer la elecci\'{o}n m\'{a}s sencilla y eficiente posible,
	$r=0\text{ o }1$:
	\begin{itemize}
		\item si $2\mid n=[F:\bb Q]$, tomar
			$\Ram(B)=\lugares[\infty]F$;
		\item si $2\nmid n$, tomar
			$\Ram(B)=\lugares[\infty]F\setmin\{v_1\}$ ($n>2$).
	\end{itemize}
	%
	En estos casos, $\spitzH2{\frak N}\simeq\spitzH[B]2{\frak N}$.
\end{frame}
