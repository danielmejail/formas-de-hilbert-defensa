\theoremstyle{remark}

%-------------

\begin{frame}{M\'{e}todo definido}
	$F=\bb Q(\sqrt 5)$, $B=\quatalg[F]{-1,-1}$,
	$\Ram(B)=\lugares[\infty]{F}$.
	\begin{itemize}
		\item Si $\cal O_0(1)$ es maximal, $H(\cal O_0(1))=1$ y
			$\dim{\spitzH[B]2{1}}=0$ (sobre $\bb Q$,
			$\spitz4{1}=0$);
		\item sobre $F$, $11=\frak n_1\frak n_2$ y
			\begin{math}
				\spitzH[B]2{\frak n_1}=0=
					\spitzH[B]2{\frak n_2}
			\end{math}, tambi\'{e}n;
		\item pero $H(\cal O_0(\frak n_1^2))=3$ y
			\begin{math}
				\spitzH[B]2{\frak n_1^2}=
					\spitzH[B]2{\frak n_1^2}^\neue\not =0
			\end{math}.
	\end{itemize}
\end{frame}

\begin{frame}{M\'{e}todo definido}
	\begin{align*}
		\dim{\spitzH[B]2{\frak n_1^2}} & \,=\,2
	\end{align*}
	%
	\begin{center}
		\begin{tabular}{c|ccccccccc}%cc}
			$\idnorm\,\frak p$ & $4$ & $5$ & $9$ &
				% $11$ & $11$ &
				$19$ & $19$ & $29$ & $29$ & $31$ & $31$ \\
			\hline
			$a_{\frak p}(f)$ & $t$ & $-t$ & $-1$ &
				% $-2t$ & $0$ &
				$4t$ & $-2$ & $-3$ & $-5t$ & $2t$ & $2t$
		\end{tabular}
	\end{center}
	($t^2-3=0$).
	\begin{align*}
		T_{\frak p_4} \,=\,
			\begin{bmatrix}
				1 & -1 \\ -2 & -1
			\end{bmatrix} & \quad\text{,}\quad
		T_{\frak p_5} \,=\,
			\begin{bmatrix}
				-1 & 1 \\ 2 & 1
			\end{bmatrix} \quad\text{,}\quad
		T_{\frak p_9} \,=\,
			\begin{bmatrix}
				-1 & 0 \\ 0 & -1
			\end{bmatrix}
	\end{align*}
	%
\end{frame}

\begin{frame}{M\'{e}todo definido}
	\begin{align*}
		\dim{\spitzH[B]2{11}} & \,=\,\dim{\spitzH[B]2{11}^\neue}\,=\,3
	\end{align*}
	%
	\begin{center}
		\begin{tabular}{c|ccccccc}%cc}%cc}
			$\idnorm\,\frak p$ & $4$ & $5$ & $9$ &
				% $11$ & $11$ &
				$19$ & $19$ & $29$ & $29$ \\
				% & $31$ & $31$ \\
			\hline
			$a_{\frak p}(f)$ & $0$ & $1$ & $-5$ &
				% $1$ & $1$ &
				$0$ & $0$ & $0$ & $0$ \\
				% & $7$ & $7$ \\
			$a_{\frak p}(g)$ & $t$ & $-t-1$ & $-t+3$ &
				% $-1$ & $-1$ &
				$-4$ & $-4$ & $2t-4$ & $2t-4$
				% & $-t+1$ & $-t+1$
		\end{tabular}
	\end{center}
	($t^2-t-8=0$).\hfill
	% \begin{flushright}
		\begin{tabular}{cc}
			$31$ & $31$ \\
			\hline
			$7$ & $7$ \\
			$-t+1$ & $-t+1$
		\end{tabular}
	% \end{flushright}
	\begin{align*}
		T_{\frak p_4} \,=\,
			\begin{bmatrix}
				0 & 0 & 0 \\ 0 & 4 & -4 \\ 5 & 1 & -3
			\end{bmatrix} & \quad\text{,}\quad
		T_{\frak p_5} \,=\,
			\begin{bmatrix}
				1 & 0 & 0 \\ 5 & -5 & 4 \\ 0 & -1 & 2
			\end{bmatrix}
	\end{align*}
	%
\end{frame}

\begin{frame}{M\'{e}todo indefinido}
	$F=\bb Q(w)$, % donde $w^3 - w^2 - 8\,w + 7=0$.
	$B=\quatalg[F]{5-w^2,1-w^2}$ ramifica en dos de los tres lugares
	reales. Sea $\cal O$ un orden de Eichler de nivel $\frak n$.
	\begin{itemize}
		\item Si $\pClass{F}=\{1,\frak a\}$, existe
			$J_{\frak a}\in\ideales{\cal O}$ con
			$\nrd(J_{\frak a})=\frak a$.
		\item $\cal O_1=\cal O$, $\cal O_{\frak a}=\Oizq(J_{\frak a})$.
		\item $\Gamma_1=\inc[\infty](\cal O_{1,+}^\times)$,
			\begin{math}
				\Gamma_{\frak a}=
					\inc[\infty](\cal O_{\frak a,+}^\times)
			\end{math} act\'{u}an en $\hP$.
	\end{itemize}
	\begin{align*}
		X_0^B(\frak n) & \,=\,\Gamma_1\backslash\hP
			\,\sqcup\,\Gamma_{\frak a}\backslash\hP
		\text{ .}
	\end{align*}
	%
	Por Eichler-Shimura,
	\begin{align*}
		\spitzH[B]2{\frak n}\,\oplus\,\lconj{\spitzH[B]2{\frak n}}
			& \,=\,\mathsf H^1\big(X_0^B(\frak n),\bb C\big)
			\,=\,\mathsf H^1\big(X(\Gamma_1),\bb C\big)\,\oplus\,
				\mathsf H^1\big(X(\Gamma_{\frak a}),\bb C\big)
		\text{ .}
	\end{align*}
	%
\end{frame}

\begin{frame}{M\'{e}todo indefinido}
	Sobre $F$, $31=\frak n_1\frak n_2$, con $\idnorm(\frak n_1)=31$,
	$\idnorm(\frak n_2)=31^2$.
	\begin{align*}
		\dim{\spitzH[B]2{\frak n_1}} \,=\, 86 & \quad\text{,}\quad
		\dim{\spitzH[B]2{\frak n_2}} \,=\, 2722 \text{ ,} \\
		\dim{\spitzH[B]2{31}^\neue} & \,=\, 81602 \text{ .}
	\end{align*}
	%
	$\spitzH[B]2{\frak n_1}$ se descompone como suma de subespacios
	Hecke-irreducibles de dimensiones $1$, $1$, $2$, $2$, $2$, $8$, $24$ y
	$46$.
	\begin{center}
		\begin{tabular}{c|ccccccccc}
			$\idnorm\,\frak p$ & $5$ & $7$ &
				$8$ & $11$ & $13$ & $17$ & $23$ & $23$ & $23$
					\\
			\hline
			$a_{\frak p}(f_1)$ & $-3$ & $4$ &
				$3$ & $0$ & $2$ & $-3$ & $5$ & $-8$ & $3$ \\
			$a_{\frak p}(f_2)$ & $-3$ & $-4$ &
				$3$ & $0$ & $-2$ & $3$ & $-5$ & $-8$ & $-3$
		\end{tabular}
	\end{center}
\end{frame}
