
\documentclass{beamer}
\mode<presentation>
{
  \usetheme{Singapore}

  \setbeamercovered{transparent}
}

\usepackage[utf8]{inputenc}
\usepackage{tikz-cd}
\usetikzlibrary{matrix}
\usetikzlibrary{decorations.pathmorphing}
\usepackage{abreviaciones}

\title{C\'{a}lculo de formas de Hilbert}
\subtitle{Formas de Hilbert y d\'{o}nde encontrarlas}
\author{Mejail, Daniel}
\institute{
	Departamento de Matem\'{a}tica\\
	Facultad de Ciencias Exactas y Naturales\\
	Universidad de Buenos Aires
}
\date{}
\subject{Talks}

% \pgfdeclareimage[height=0.5cm]{university-logo}{university-logo-filename}
% \logo{\pgfuseimage{university-logo}}

\AtBeginSubsection[]
{
  \begin{frame}<beamer>{Contenidos}
    \tableofcontents[currentsection,currentsubsection]
  \end{frame}
}

% \beamerdefaultoverlayspecification{<+->}

\setbeamertemplate{itemize subitem}{$\circ$}

\newcounter{saveenumi}
\newcommand{\seti}{\setcounter{saveenumi}{\value{enumi}}}
\newcommand{\conti}{\setcounter{enumi}{\value{saveenumi}}}
\resetcounteronoverlays{saveenumi}

\begin{document}

\begin{frame}
  \titlepage
\end{frame}

\begin{frame}{Contenidos}
  \tableofcontents
  %% You might wish to add the option [pausesections]
  % Vamos a empezar repasando la definici\'{o}n de una forma modular en una sola variable,
  % espec\'{\i}ficamente, formas modulares para el grupo modular $\SL[2](\bb{Z})$, para
  % pasar, luego, a la definici\'{o}n de formas de Hilbert (en un caso especial), haciendo una
  % analog\'{\i}a con el caso el\'{\i}ptico.
  %
  % Una vez que hayamos mencionado las propiedades b\'{a}sicas de estos objetos,
  % trataremos de entender qu\'{e} significa calcularlos, es decir, qu\'{e} informaci\'{o}n
  % deseamos obtener acerca de ellos; no estaremos buscando una f\'{o}rmula expl\'{\i}cita
  % para las funciones que los representan, como tampoco una representaci\'{o}n de todas
  % las posibles formas. Las formas de Hilbert constituyen espacios vectoriales, as\'{\i} que el
  % problema de calcular estos espacios es hallar bases para cada uno de ellos. La soluci\'{o}n
  % de este problema requiere una reducci\'{o}n para saber con qu\'{e} nos podemos contentar.
  %
  % A continuaci\'{o}n, introducimos las formas modulares cuaterni\'{o}nicas y enunciamos la
  % correspondencia de J-L, la relaci\'{o}n entre las formas cuaterni\'{o}nicas con las formas de
  % Hilbert. La existencia de esta correspondencia da lugar a una familia de m\'{e}todos para
  % calcular formas de Hilbert.
  %
  % Finalmente, terminamos con algunos ejemplos ilustrando esstos m\'{e}todos.
  %
\end{frame}

\section{Formas de Hilbert}\label{sec:hilbert}
\theoremstyle{remark}
\newtheorem{teoMultUno}{Teorema}[section]
\newtheorem{teoPrincipioKoecher}[teoMultUno]{Teorema}
\newtheorem{defFormaEliptica}{Definici\'{o}n}[section]
\newtheorem{obsSimplificacionCuadraticoReal}[teoMultUno]%
	{Simplificaci\'{o}n}
\newtheorem{obsSubgruposCongruencia}[teoMultUno]{Observaciones}
\newtheorem{defFormaHilbertClasica}[teoMultUno]{Definici\'{o}n}
\newtheorem{obsDesarrolloDeFou}[teoMultUno]{Observaci\'{o}n}
\newtheorem{defNotacion}[teoMultUno]{Definici\'{o}n}
%
\newtheorem{defFormasNuevas}[teoMultUno]{Definici\'{o}n}
\newtheorem{coroBaseFormasNuevas}[teoMultUno]{Corolario}

%-------------

\subsection{Formas modulares: repaso}

\begin{frame}{El semiplano complejo superior}
	% Empezamos situ\'{a}ndonos en un espacio conocido:
	El semiplano complejo superior es
	\begin{align*}
		\hP & \,=\,\big\{z\in\bb{C}\,:\,\Im(z)>0\big\}
		\text{ .}
	\end{align*}
	%
	% y un grupo actuando en $\hP$
	El \emph{grupo modular},
	\begin{align*}
		\SL[2](\bb{Z}) & \,=\,\bigg\{
			\begin{bmatrix} a & b \\ c & d \end{bmatrix} \,:\,
			a,b,c,d\in\bb{Z},\,ad-bc=1\bigg\}% \\
		\text{ ,}
	\end{align*}
	%
	act\'{u}a en $\hP$ v\'{\i}a
	% El grupo $\SL[2](\bb R)$ act\'{u}a transitivamente v\'{\i}a
	\begin{align*}
		\begin{bmatrix} a & b \\ c & d \end{bmatrix}\cdot z & \,=\,
			\frac{az+b}{cz+d}
		\text{ .}
	\end{align*}
	%
\end{frame}

\begin{frame}{El grupo modular}
	$\Gamma:=\SL[2](\bb Z)$
	\begin{itemize}
		\item El cociente $Y(1)=\Gamma\backslash\hP$ parametriza clases
			de isomorfismo de curvas el\'{\i}pticas sobre $\bb{C}$:
			\begin{align*}
				\tau\in\hP\mapsto E_\tau & \quad\text{,}\quad
				\gamma\tau=\tau'\,\Rightarrow\,
					E_\tau\simeq E_{\tau'}
			\end{align*}
			%
	% --curvas complejas con estructura de grupo (abeliano) compatible:
	% $Y(1)$ es una curva compleja no compacta, se puede compactificar
	% agregando un \'{u}nico punto adicional, $\infty$, la ``c\'{u}spide''
	% en infinito.
		\item La curva compleja $Y(1)$ admite una compactificaci\'{o}n
			agregando una \emph{c\'{u}spide}:
			$X(1)=Y(1)\cup\{\infty\}$
	% Las funciones (meromorfas, holomorfas) en el cociente, est\'{a}n en
	% biyecci\'{o}n con las funciones en el semiplano (meromorfas,
	% holomorfas) que son invariantes por la acci\'{o}n del grupo modular.
		\item Existe una biyecci\'{o}n
			\begin{align*}
				\Big\{f:\,\Gamma\backslash\hP\rightarrow\bb{C}
					\Big\} & \,\leftrightarrow\,
				\Big\{f:\,\hP\rightarrow\bb{C}\,:\,
					f(\gamma z)=f(z)\,
					\forall\gamma\in\Gamma\Big\}
			\end{align*}
			%
	\end{itemize}
	%
	% Adem\'{a}s, las funciones meromorfas u holomorfas en $X(1)$ se
	% corresponden con funciones meromorfas u holomorfas, invariantes y que
	% satisfacen una condici\'{o}n de crecimiento a medida que el argumento
	% tiende al borde del semiplano. A tales funciones las llamamos
	% funciones modulares.
\end{frame}

\begin{frame}{Formas modulares para $\SL[2](\bb{Z})$}
	% Ahora bien, como $X(1)$ es una superficie de Riemann compacta, no
	% existen funciones holomorfas en $X(1)$, es decir, holomorfas en
	% $Y(1)$ que se extiendan a una funci\'{o}n holomorfa en la
	% compactificaci\'{o}n. Pero s\'{\i} existen funciones holomorfas, que
	% verifican una regla de transformaci\'{o}n en relaci\'{o}n con la
	% acci\'{o}n del grupo modular y que son, en cierto sentido, holomorfas
	% en las c\'{u}spides, en $\infty$.
	\begin{defFormaEliptica}\label{def:formaelipticacompleto}
		Dado $k\in\bb{Z}$, decimos que $f:\,\hP\rightarrow\bb{C}$ es
		una \emph{forma modular de peso $k$ para $\SL[2](\bb Z)$}
		($f\in\modular{k}{\SL[2](\bb Z)}$), si:
		\begin{enumerate}
			\item\label{def:formaelipticacompleto:holomorfia}
				$f$ es holomorfa en $\hP$,
			\item\label{def:formaelipticacompleto:invarianza}
				para toda
				\begin{math}
					\left[\begin{smallmatrix}
						a & b \\ c & d
					\end{smallmatrix}\right]\in
						\SL[2](\bb Z)
				\end{math}
				\begin{align*}
					f\bigg(\dfrac{az+b}{cz+d}\bigg) & \,=\,
						(cz+d)^{k}\,f(z)
					\quad\text{y}
				\end{align*}
			\item\label{def:formaelipticacompleto:cuspides}
				$f$ es holomorfa en $\infty$;
			\seti
		\end{enumerate}
		%
		una forma $f\in\modular{k}{\SL[2](\bb Z)}$ se dice
		\emph{cuspidal} ($f\in\spitz{k}{\SL[2](\bb Z)}$), si,
		adem\'{a}s,
		\begin{enumerate}
			\conti
			\item\label{def:formaelipticacompleto:cuspidal}
				$f$ se anula en $\infty$.
			\seti
		\end{enumerate}
		%
	\end{defFormaEliptica}
\end{frame}

\begin{frame}{Desarrollo de Fourier}
	Si $f$ verifica \ref{def:formaelipticacompleto:holomorfia} y
	\ref{def:formaelipticacompleto:invarianza}, entonces posee un
	\emph{desarrollo de Fourier}:
	\begin{align*}
		f(z) & \,=\,\sum_{n\in\bb{Z}}\,a_{n}(f)\,e^{2\pi i z n}
		\text{ .}
	\end{align*}
	%
	\begin{itemize}
		\item $f\in\modular{k}{\SL[2](\bb{Z})}$, si $a_{n}(f)=0$,
			siempre que $n<0$, y
		\item $f\in\spitz{k}{\SL[2](\bb{Z})}$, si, adem\'{a}s,
			$a_{0}(f)=0$.
	\end{itemize}
	%
	% Equivalentemente, estas condiciones de holomorf\'{\i}a y
	% anulaci\'{o}n, se pueden expresar en t\'{e}rminos del orden de
	% crecimiento de $f$ a medida que el argumento tiende a $\infty$ y al
	% borde de $\hP$
\end{frame}

\begin{frame}{Otros grupos}
	El grupo $\SL[2](\bb R)$ act\'{u}a transitivamente en $\hP$. Podemos
	obtener un conciente $\Gamma\backslash\hP$ a partir de
	$\Gamma\subset\SL[2](\bb R)$ bajo ciertas condiciones:
	% Esta generalidad adicional ser\'{a} \'{u}til m\'{a}s adelante, en
	% relaci\'{o}n con grupos provenientes de \'{a}lgebras de cuaterniones.
	\begin{itemize}
		\item $\SL[2](\bb{Z})$ y subgrupos de congruencia, por ejemplo
			\begin{align*}
				\Gamma_{0}(N) & \,=\,\bigg\{
					\begin{bmatrix} a & b \\
					c & d \end{bmatrix}\in\SL[2](\bb{Z})
					\,:\,c\equiv 0\,(\modulo\,N)\bigg\}
				\text{ ;}
			\end{align*}
			%
		\item grupos de unidades de \'{o}rdenes en \'{a}lgebras de
			cuaterniones indefinidas (m\'{a}s adelante).
	\end{itemize}
	%
\end{frame}


\subsection{Formas de Hilbert}

\begin{frame}{Algunos preliminares}
	Sea $F=\bb Q(\sqrt d)$, $d>0$. Existen dos inclusiones
	$F\hookrightarrow\bb R$:
	% Podemos mirar $F$ dentro de $\bb R$ de dos maneras:
	\begin{align*}
		\tau_1:\,a+b\sqrt d\mapsto a+b\sqrt d & \quad\text{y}\quad
			\tau_2:\,a+b\sqrt d\mapsto a-b\sqrt d
		\text{ .}
	\end{align*}
	%
	Un elemento $x\in F$ se dice \emph{totalmente positivo} (escribimos
	$x\gg 0$ o $x\in F^{\times}_{+}$), si $x_{j}=\tau_j(x)>0$ para $j=1,2$.
	\begin{defNotacion}\label{def:notaciontotalmentereal}
		\begin{itemize}
			\item $\oka{F}$: anillo de enteros de $F$.
			\item $\Class{F}$: \emph{grupo de clases} de $F$
				($\frak{a}\sim\frak{b}$, si
				$\frak{a}=\lambda\frak{b}$,
				$\lambda\in F^{\times}$).
			\item $\pClass{F}$: \emph{grupo de clases estrictas}
				de $F$ ($\lambda\gg0$).
			\item $\lugares{F}$: conjunto de lugares de  $F$
				(primos $\frak p\subset \oka F$,
				$|\cdot|_{\tau_1}$, $|\cdot|_{\tau_2}$).
		\end{itemize}
	\end{defNotacion}
\end{frame}

\begin{frame}{Ejemplos}
	\begin{itemize}
		\item $F=\bb{Q}(\sqrt{d})$, $d>0$,
			\begin{align*}
				\oka F & \,=\,
				\begin{cases}
					\Big\{a+b\tfrac{1+\sqrt d}{2}\,:\,
						a,b\in\bb Z\Big\}
						& d\equiv 1\,(4) \\[5pt]
					\Big\{a+b\sqrt d\,:\,a,b\in\bb Z\Big\}
						& d\equiv2,3\,(4)
				\end{cases}
				\text{ ;}
			\end{align*}
			%
		\item $F=\bb{Q}(w)$, $w^3 - w^2 - 8\,w + 7=0$,
			$\oka F=\bb Z[w]$.
	\end{itemize}
	%
	\begin{center}
	\begin{tabular}{c|c|c}
		& $h$ & $h^{+}$ \\
		\hline
		$\bb{Q}(\sqrt{5})$ & $1$ & $1$ \\
		$\bb{Q}(w)$ & $1$ & $2$ \\
		$\bb{Q}(\sqrt{10})$ & $2$ & $2$
	\end{tabular}
	\begin{tabular}{c}
		\vphantom{$h^{+}$} \\
		%
		\vphantom{$\bb Q(\sqrt 5)$} \\
		\vphantom{$\bb Q(w)$}
			\begin{math}
				\rightarrow \pClass{\bb Q(w)} =
				\{\generado{1},\generado{w^2 -4}\}
			\end{math} \\
		\vphantom{$\bb Q(\sqrt 10)$}
	\end{tabular}
	\end{center}
	% En general,
	% $h^+=h\Leftrightarrow \oka{F,+}^\times=(\oka F^\times)^2$.
\end{frame}

\begin{frame}{Acci\'{o}n en $\hP^{2}$}
	% Las formas modulares de Hilbert se pueden ver como generalizaciones a
	% cuerpos totalmente reales de la noci\'{o}n de forma modular
	% el\'{\i}ptica. 
	%
	% Si queremos extender la definici\'{o}n \ref{def:formaeliptica},
	% necesitamos contar con un espacio $H$ que juegue el rol de $\hP$ y con
	% un grupo $G(F)$ que act\'{u}e en $H$ y que tenga algo que ver con $F$.
	Si $(z_1,z_2)\in\hP^2$ y
	\begin{math}
		\left[\begin{smallmatrix}
			a & b \\ c & d
		\end{smallmatrix}\right]\in\GLtp[2](F)=
			\Big\{\gamma\in\GL[2](F)\,:\,\det(\gamma)\gg 0\Big\}
	\end{math},
	\begin{align*}
		\begin{bmatrix} a & b \\ c & d \end{bmatrix}\cdot (z_1,z_2)
			& \,:=\,\bigg(
			\frac{a_{1}z_{1}+b_{1}}{c_{1}z_{1}+d_{1}},
			\frac{a_{2}z_{2}+b_{2}}{c_{2}z_{2}+d_{2}}\bigg)
		\text{ .}
	\end{align*}
	%
	Dadas $f:\,\hP^2\rightarrow\bb{C}$ y
	\begin{math}
		\gamma= \left[\begin{smallmatrix}
			a & b \\ c & d
			\end{smallmatrix}\right]\in\GLtp[2](F)
	\end{math},
	\begin{align*}
		f\operadormatrices{\null}{\gamma}(z_1,z_2) & \,:=\,
			\frac{\det(\gamma_1)}{(c_1z_1+d_1)^{2}}\,
			\frac{\det(\gamma_2)}{(c_2z_2+d_2)^{2}}\,
			f(\gamma\cdot (z_1,z_2))
		\text{ .}
	\end{align*}
	%

\end{frame}

\begin{frame}{Subgrupos de congruencia}
	% El grupo modular admite un an\'{a}logo sobre un cuerpo totalmente
	% real $F$: $\SL[2](\oka F)$, el \emph{grupo modular de Hilbert}.
	%
	% El grupo de unidades $\oka{F}^{\times}$ (y, en especial, el subgrupo
	% de unidades totalmente positivas, $\oka{F,+}^{\times}$) juega un rol
	% importante en la teor\'{\i}a aritm\'{e}tica de las formas de Hilbert,
	% con lo cual consideramos, de entrada, los grupos
	Si $F=\bb Q(\sqrt d)$,
	$\oka F^\times=\big\{\pm\epsilon_0^k\,:\,k\in\bb Z\big\}$. Definimos
	\begin{align*}
		\GLtp[2](\oka{F}) & \,:=\,\bigg\{
			\begin{bmatrix} a & b \\ c & d \end{bmatrix} \,:\,
				a,b,c,d\in\oka{F},\,ad-bc\in\oka{F,+}^{\times}
				\bigg\} \\
		\Gamma_{0}(\frak{N}) & \,:=\, \bigg\{
			\begin{bmatrix} a & b \\ c & d \end{bmatrix}\in
				\GLtp[2](\oka{F}) \,:\,c\in\frak{N}\bigg\}
			\qquad\text{(\phantom)}\frak N\subset\oka F
				\text{\phantom().}
	\end{align*}
	%
	\begin{obsSubgruposCongruencia}\label{obs:subgruposcongruencia}
		\begin{itemize}
			\item Si $\epsilon\in\oka{F,+}^\times$ y
				$\mu\in\oka F$,
				\begin{math}
					\left[\begin{smallmatrix}
						\epsilon & \mu\vphantom{1} \\
						& 1
					\end{smallmatrix}\right]\in
					\Gamma_0(\frak N)_\infty
				\end{math};
			\item $\Gamma_0(\frak N)\backslash\hP^2$ es una
				superficie no compacta (finitas c\'{u}spides);
			\item asumimos que $h^+(F)=1$.
		\end{itemize}
	\end{obsSubgruposCongruencia}
\end{frame}

\begin{frame}{Formas modulares para $\Gamma_{0}(\frak{N})$}
	\begin{defFormaHilbertClasica}\label{def:formahilbertclasica}
		Una funci\'{o}n $f:\,\hP^2\rightarrow\bb{C}$ es una
		\emph{forma modular de peso $(2,2)$ para %
		$\Gamma_{0}(\frak{N})$} ($f\in\modularH{2}{\frak{N}}$), si:
		\begin{enumerate}
			\item\label{def:formahilbertclasica:holomorfia}
				$f$ es holomorfa en $\hP^2$;
			\item\label{def:formahilbertclasica:invarianza}
				para toda
				$\gamma\in\Gamma_{0}(\frak{N})$,
				\begin{math}
					f\operadormatrices{\null}{\gamma}=f
				\end{math}.
			\seti
		\end{enumerate}
	\end{defFormaHilbertClasica}
	\begin{obsDesarrolloDeFou}[Desarrollo de Fourier]%
		\label{obs:desarrollodefou}
		Por \ref{def:formahilbertclasica:invarianza}, $f(z+\mu)=f(z)$
		para todo $\mu\in\oka F$ y
		\begin{align*}
			f & \,=\,\sum_{\nu\in\oka F^{\perp}}\,a_{\nu}(f)\,
				e^{2\pi i \traza(\nu z)}
			\text{ ,}
		\end{align*}
		%
		donde $\traza(\nu z)=\nu_1z_1+\nu_2z_2$ y
		\begin{math}
			\oka F^{\perp}=\big\{
				\nu\in F\,:\,\traza(\nu\oka F)\subset\bb{Z}
				\big\}
		\end{math}.
	\end{obsDesarrolloDeFou}
\end{frame}

\begin{frame}{Principio de Koecher y formas cuspidales}
	\begin{teoPrincipioKoecher}\label{thm:principiokoecher}
		\begin{itemize}
			\item $a_{\epsilon\nu}= a_{\nu}$ para todo
				$\nu\in\oka F^{\perp}$,
				$\epsilon\in\oka{F,+}^\times$;
			\item si $a_{\nu}\not=0$, entonces $\nu=0$ o
				$\nu\gg 0$.
		\end{itemize}
	\end{teoPrincipioKoecher}

	Para ver lo que pasa en otra c\'{u}spide, $x\in\bb{P}^{1}(F)$, miramos
	el desarrollo de $f\operadormatrices{\null}{A}$, con
	$A\cdot\infty=x$.
	% Toda funci\'{o}n holomorfa y de peso $\peso{k}$ invariante para
	% $\Gamma_{0}(\frak{N},\frak{a})$ es holomorfa en las c\'{u}spides.
	\begin{defFormaHilbertClasica}\label{def:formahilbertclasicacuspidal}
		Una forma $f$ se dice \emph{cuspidal}
		($f\in\spitzH{2}{\frak{N}}$), si, adem\'{a}s,
		\begin{enumerate}
			\conti
			\item\label{def:formahilbertclasica:cuspidal}
				\begin{math}
					a_{0}\big(f\operadormatrices{%
						\null}{A}\big)=0
				\end{math} para toda $A\in\GLtp[2](F)$.
			\seti
		\end{enumerate}
		%
	\end{defFormaHilbertClasica}
\end{frame}

\begin{frame}{Operadores de Hecke y coeficientes de Fourier}
	El espacio $\spitzH2{\frak N}$ posee un producto interno y, para cada
	primo $\frak p\nmid\frak N$, operadores
	\begin{math}
		T_{\frak p}:\,\spitzH2{\frak N}\rightarrow \spitzH2{\frak N}
	\end{math} tales que
	\begin{itemize}
		\item $T_{\frak p}T_{\frak q}=T_{\frak q}T_{\frak p}$ y
		\item
			\begin{math}
				\langle T_{\frak p}f,g\rangle=
					\langle f,T_{\frak p}g\rangle
			\end{math}.
	\end{itemize}
	%
	Dado $\frak{m}\subset\oka{F}$, existe $\nu\in\diferente^{-1}$,
	$\nu\gg0$ tal que $\frak{m}=\nu\,\diferente$.
	\begin{align*}
		C(\frak m,f) & \,:=\,\idnorm(\frak m)\,a_\nu(f)
		\text{ .} \\
		C(\frak m,T_{\frak p}f) & \,=\,
			\idnorm(\frak p)\,C(\frak{p}^{-1}\frak m,f)\,+\,
				C(\frak p\frak m,f)
		\text{ .}
	\end{align*}
	%
	($\frak p\nmid\frak m\Rightarrow C(\frak{p}^{-1}\frak m,f)=0$).
\end{frame}

\begin{frame}{El espacio de formas nuevas}
	\begin{defFormasNuevas}\label{def:formasnuevas}
		Si $\frak N=\frak l\,\frak M$, $\frak l$ primo, existen
		\begin{math}
			\iota_1,\iota_\frak{l}:\,\spitzH{2}{\frak{M}}
				\hookrightarrow\spitzH{2}{\frak{N}}
		\end{math}.
		% El \emph{subespacio de formas nuevas en $\frak l$} en
		% $\spitzH2{\frak N}$ est\'{a} dado por el complemento
		% ortogonal respecto del p.i. de Petersson,
		\begin{align*}
			\spitzH2{\frak N}^{\frak l-\neue} & \,:=\,
				\Big(
				\iota_1\big(\spitzH2{\frak M}\big)\,+\,
				\iota_\frak{l}\big(\spitzH2{\frak M}\big)
					\Big)^\perp
				\text{ ,} \\
			\spitzH2{\frak N}^\neue & \,:=\,
				\bigcap_{\frak l\mid\frak N}\,
					\spitzH2{\frak N}^{\frak l-\neue}\,=\,
				\Big(\spitzH2{\frak N}^\oude\Big)^\perp
			\text{ .}
		\end{align*}
		%
	\end{defFormasNuevas}
	Son espacios $T_{\frak p}$-invariantes ($\frak p\nmid\frak N$).
	\begin{coroBaseFormasNuevas}\label{coro:baseformasnuevas}
		Existe una base ortogonal para $\spitzH2{\frak N}^\neue$
		compuesta por autoformas para $T_{\frak p}$,
		$\frak p\nmid\frak N$ (\emph{formas nuevas}).
	\end{coroBaseFormasNuevas}
\end{frame}

\begin{frame}{Formas nuevas}
	Si $f$ es autoforma,
	\begin{align*}
		C(\frak p,f) & \,=\,C(\oka F,T_{\frak p}f)\,=\,
			\lambda_{\frak p}\,C(\oka F,f)
			\qquad\text{(\phantom)}\frak p\nmid\frak N
			\text{\phantom().}
	\end{align*}
	%
	Si $C(\oka F,f)\not=0$, podemos asumir $C(\oka F,f)=1$ (normalizada).

	\begin{teoMultUno}[Multiplicidad uno]\label{coro:multuno}
		Dado un ideal \'{\i}ntegro $\frak N$ y dado un sistema (de
		autovalores)
		\begin{math}
			\{\lambda_{\frak p}\}_{\frak p\nmid\frak N}\subset
				\bb{C}
		\end{math}, existe, a lo sumo, una forma nueva normalizada $f$
		de nivel $\frak M\mid\frak N$ tal que
		$C(\frak p,f)=\lambda_{\frak p}$ para todo
		$\frak p\nmid\frak N$.
	\end{teoMultUno}
\end{frame}

\begin{frame}{Calculando $\spitzH2{\frak N}$}
	\begin{itemize}
		\item Objetivo: hallar una base para el espacio
			$\spitzH{2}{\frak{N}}$.
			% $\spitzH{2}{\frak{N}}$ es un $\bb{C}$-e.v., as\'{\i}
			% que, en definitiva, lo que vamos a buscar es una base
			% para este espacio.
		\item Si $F=\bb Q$, formas modulares el\'{\i}pticas:
			% Como ocurre en general, si queremos resolver este
			% problema de manera computacional, vamos a necesitar
			% contar con una realizaci\'{o}n concreta de este
			% espacio vectorial, o, mejor dicho, una
			% representaci\'{o}n equivalente de los operadores de
			% Hecke.
			\begin{itemize}
				\item el espacio $\spitz{2}{N}$
					% de formas cuspidales el\'{\i}pticas
					se realiza en la cohomolog\'{\i}a de
					% la curva modular
					$X_0(N)$ (Eichler-Shimura);
				\item si $N$ no es un cuadrado, las formas en
					$\spitz{2}{N}^{\neue}$ aparecen como
					combinaciones lineales de series
					\emph{theta}.
			\end{itemize}
			%
		\item Si $[F:\bb Q]=n$, $f\in\spitzH2{\frak N}$ da lugar a una
			$n$-forma diferencial holomorfa
			$f(z_1,\,\dots,\,z_n)\,dz_1\,\cdots\,dz_n$ en
			$X_0(\frak N)$.
		\item $n>1$ es un problema dif\'{\i}cil
			?`Hay otra manera?%de realizar este espacio?
	\end{itemize}
	%
	La correspondencia de Jacquet-Langlands garantiza que estas formas
	tambi\'{e}n se pueden encontrar en la cohomolog\'{\i}a de otras
	variedades. En todo caso, alcanza con mirar variedades de dimensi\'{o}n
	$0$ o $1$.
\end{frame}


\section{Formas modulares cuaterni\'{o}nicas}\label{sec:cuaternionicas}
\theoremstyle{remark}
\newtheorem{teoNorma}{Teorema}[section]
\newtheorem{defAlgebraDeCuaterniones}[teoNorma]{Definici\'{o}n}
\newtheorem{ejemMatrices}[teoNorma]{Ejemplos}
\newtheorem{teoClasificacion}[teoNorma]{Teorema}
\newtheorem{obsExtension}[teoNorma]{Observaci\'{o}n}
\newtheorem{defRamifica}[teoNorma]{Definici\'{o}n}
\newtheorem{defOrdenesEIdeales}[teoNorma]{Definiciones}
\newtheorem{lemaDiscriminanteReducido}[teoNorma]{Lema}
\newtheorem{obsDiscriminanteReducido}[teoNorma]{Observaci\'{o}n}
\newtheorem{defOrdenDeEichler}[teoNorma]{Definici\'{o}n}
\newtheorem{defCuaternionicas}[teoNorma]{Definici\'{o}n}
\newtheorem{obsCuaternionicas}[teoNorma]{Observaciones}
\newtheorem{teoCorrespondenciaJL}[teoNorma]{Teorema}
\newtheorem{obsCorrespondenciaJL}[teoNorma]{Observaci\'{o}n}

%-------------

\subsection{\'{A}lgebras de cuaterniones}\label{subsec:preliminares}

\begin{frame}{\'{A}lgebras de cuaterniones}
	\begin{defAlgebraDeCuaterniones}\label{def:algebradecuaterniones}
		Un \'{a}lgebra de cuaterniones sobre un cuerpo $F$
		($\car(F)\not=2$) es una $F$-\'{a}lgebra $B$ con dos
		generadores $i,j$ que verifican
		\begin{align*}
			i^2\,=\,a & \text{ ,}\quad j^2\,=\,b
				\quad\text{y}\quad ji\,=\,-ij
		\end{align*}
		%
		$a,b\in F^\times$, $k:=ij$, $B=\quatalg[F]{a,b}$. Si $h\in B$,
		es ra\'{\i}z de
		\begin{align*}
			X^2\,-\,\trd(h)\,X\,+\,\nrd(h) & \,=\,
				X^2\,-\,(h+\conj h)\,X\,+\,\conj hh
			\text{ .}
		\end{align*}
		%
	\end{defAlgebraDeCuaterniones}
	\begin{ejemMatrices}\label{ejem:matrices}
		\begin{itemize}
			\item $\MM[2\times 2](\bb Q)$:
				\begin{math}
					i=\left[\begin{smallmatrix}
						& 1 \\ 1 &
					\end{smallmatrix}\right]
				\end{math},
				\begin{math}
					j=\left[\begin{smallmatrix}
						-1 & \\ & 1
					\end{smallmatrix}\right]
				\end{math}, $\nrd=\det$ y
				$\trd=\traza$;
			\item $\quatalg[\bb Q]{-1,-1}\subset\bb H$.
		\end{itemize}
	\end{ejemMatrices}
\end{frame}

\begin{frame}{\'{A}lgebras de cuaterniones (cont.)}
	\begin{obsExtension}\label{obs:extension}
		Dadas $B/F$ y una extensi\'{o}n $K/F$, $B\otimes_F K$ es un
		\'{a}lgebra de cuaterniones sobre $K$. (%
		\begin{math}
			\quatalg[\bb Q]{-1,-1}\otimes_{\bb Q}\bb R\simeq\bb H
		\end{math}).
		%
	\end{obsExtension}

	\begin{defRamifica}\label{def:ramifica}
		\begin{itemize}
			\item\emph{$B$ ramifica en $v$}, si $B_v/\bb Q_v$ es de
				divisi\'{o}n ($v=p,\infty$);
			\item\emph{discriminante}:
				\begin{math}
					\disc[B]:=\prod_{p\in\Ram(B)}\,
						\generado{p}
				\end{math}. (%
				\begin{math}
					\disc[{\quatalg[\bb Q]{-1,-1}}]=2\bb Z
				\end{math});
			\item $B$ es \emph{indefinida}, si
				$B_\infty\simeq\MM[2\times2](\bb R)$;
			\item $B$ es \emph{definida}, si
				$B_{\infty}\simeq\bb H$.
		\end{itemize}
	\end{defRamifica}

	\begin{teoClasificacion}[Clasifici\'{o}n global]%
		\label{thm:clasificacionglobal}
		Sobre un cuerpo de n\'{u}meros,
		\begin{math}
			\big[B/F\big] \,\leftrightarrow\,
				\text{subconjuntos pares de }\lugares{F}
		\end{math}
	\end{teoClasificacion}
\end{frame}

\begin{frame}{Ejemplos}
	\begin{center}
		\begin{tabular}{c|ccc}
			$B$ & $\Ram(B)$ & $\disc[B]$ & $B_\infty$ \\
			\hline
			$\quatalg[\bb Q]{-1,-11}$ & $\{11,\infty\}$ &
				$11\cdot\bb Z$ & $\bb H$ \\
			$\quatalg[\bb Q(\sqrt 5)]{-1,-1}$ &
				$\{\tau_1,\tau_2\}$ & $1$ &
				$B_{\tau_1},B_{\tau_2}\simeq\bb H$ \\
			$\quatalg[\bb Q(w)]{5- w^2,1-w^2}$ &
				$\{\tau_1,\tau_3\}$ & $1$ &
				$B_{\tau_2}\simeq\MM[2\times 2](\bb R)$
		\end{tabular}
	\end{center}
	\begin{align*}
		& w^3 - w^2 - 8\,w + 7 \,=\,0 \\
		& \quad\tau_1 \,:\,w\,\mapsto\,-2,781\dots \text{ ,} \\
		& \quad\tau_2 \,:\,w\,\mapsto\,0,8621\dots \text{ ,} \\
		& \quad\tau_3 \,:\,w\,\mapsto\,2,919\dots
	\end{align*}
	%
\end{frame}

\begin{frame}{\'{O}rdenes e ideales}
	\begin{defOrdenesEIdeales}\label{def:ordeneseideales}
		\begin{itemize}
			\item Un \emph{ret\'{\i}culo (completo)} en $B$ es un
				$\bb Z$-m\'{o}dulo $I\subset B$ f.g. que
				contiene una base ($\bb Q\cdot I=B$);
			\item $x\in B$ es \emph{\'{\i}ntegro}, si
				$\nrd(x),\trd(x)\in\bb Z$;
			\item un \emph{orden} es un ret\'{\i}culo $\cal O$ que
				es subanillo (con $1$);
			\item
				\begin{math}
					\Oder(I):=\big\{h\in B\,:\,
						I\,h\subset I\big\}
				\end{math} es un orden;
			\item un \emph{$\cal O$-ideal (a derecha)} es un
				ret\'{\i}culo $I$ tal que $\Oder(I)=\cal O$.
		\end{itemize}
		%
	\end{defOrdenesEIdeales}
\end{frame}

% \begin{frame}{\'{O}rdenes e ideales (cont.)}
	% % Todo orden tiene asociado un ideal que mide, entre otras cosas,
	% % cu\'{a}n lejos est\'{a} el orden de ser maximal.
	% Dados $\beta_1,\,\beta_2,\,\beta_3,\,\beta_4\in B$,
	% \begin{align*}
		% \disc(\beta_1,\beta_2,\,\beta_3,\,\beta_4) & \,:=\,
			% \det\,\trd(\beta_i\beta_j)
		% \text{ .}
	% \end{align*}
	% %
	% El \emph{discriminante (reducido)} de $\cal O$ es el ideal
	% \begin{align*}
		% \drd(\cal O) & \,:=\,\sqrt{\generado{\disc(%
			% \beta_1,\,\beta_2,\,\beta_3,\,\beta_4)\,:\,%
			% \beta_i\in \cal O}}
		% \text{ .}
	% \end{align*}
	% %
	% % Esto requiere algunas explicaciones: si $\{\beta_i\}_i$ es un
	% % conjunto generador de $\cal O$ (pseudobase), entonces
	% % $\disc(\alpha_i)=\det(M)^2\,\disc(\beta_i)$, $M$ la matriz de
	% % coeficientes de los $\alpha_i$ en t\'{e}rminos de los $\beta_i$.
	% % En general, $\det(M)\in\frak a$, para cierto ideal (?`\'{\i}ntegro?)
	% % del cuerpo de base y el discriminante (no reducido) de $\cal O$
	% % es $\frak a^2\,\disc(\beta_i)$ (o algo del estilo). Adem\'{a}s,
	% % se puede ver (usando la relaci\'{o}n con el \'{\i}ndice y que existe
	% % un orden con discriminante cuadrado) que el discriminante de
	% % cualquier orden (ret\'{\i}culo) es un cuadrado. El discriminante
	% % reucido es igual al ideal que se obtiene dividiendo los exponentes.
	% %
	% \begin{lemaDiscriminanteReducido}\label{lema:discriminantereducido}
		% Si $\cal O\subset\cal O'$, entonces
		% \begin{math}
			% \drd(\cal O)\subset\drd(\cal O')
		% \end{math}.
		% En ese caso, $\cal O=\cal O'$, si y s\'{o}lo si
		% $\drd(\cal O)=\drd(\cal O')$.
	% \end{lemaDiscriminanteReducido}
% \end{frame}

\begin{frame}{\'{O}rdenes e ideales (cont.)}
	\begin{lemaDiscriminanteReducido}\label{lema:discriminantereducido}
		Todo orden $\cal O\subset B$ tiene asociado un ideal
		$\drd(\cal O)\subset F$ de manera que, si
		$\cal O\subset\cal O'$, entonces
		\begin{math}
			\drd(\cal O)\subset\drd(\cal O')
		\end{math}.
		En ese caso, $\cal O=\cal O'$, si y s\'{o}lo si
		$\drd(\cal O)=\drd(\cal O')$.
	\end{lemaDiscriminanteReducido}
	\begin{obsDiscriminanteReducido}\label{obs:discriminantereducido}
		Existen \'{o}rdenes maximales.
	\end{obsDiscriminanteReducido}
	\begin{defOrdenDeEichler}\label{def:ordendeeichler}
		Un \emph{orden de Eichler} es la intersecci\'{o}n de dos
		\'{o}rdenes maximales. Si $\cal O$ es de Eichler,
		\begin{align*}
			\drd(\cal O) & \,=\,\disc[B]\cdot\frak N
			\text{ ,}
		\end{align*}
		%
		$\frak N$ es el \emph{nivel} del orden y
		$(\disc[B],\frak N)=1$.
	\end{defOrdenDeEichler}
\end{frame}

\begin{frame}{Ejemplos}
	\begin{itemize}
		\item $B=\MM[2\times 2](F)$,
			\begin{align*}
				\cal O\,:=\,
				\left[\begin{matrix}
					\oka F & \oka F \\ \frak N & \oka F
				\end{matrix}\right] \,\subset\,
				\MM[2\times 2](\oka F)
				& \qquad \text{(\phantom)}
					\cal O_+^\times=\Gamma_0(\frak N)
					\text{\phantom();}
			\end{align*}
			%
		\item $B=\quatalg[\bb Q]{-1,-11}$, $\generado{1,i,j,k}$
			no es maximal:
			\begin{center}
				\begin{tabular}{c|c}
					$\cal O$ &
						\begin{math}
							\disc(\beta_1,\,
								\beta_2,\,
								\beta_3,\,
								\beta_4)
						\end{math} \\
					\hline
					$\generado{1,i,j,k}$ &
						$4^2\,11^2$ \\
					$\generado{1,i,j,\frac{1+i+j+k}{2}}$ &
						$2^2\,11^2$ \\
					$\generado{1,i,\frac{1+j}{2},%
						\frac{i+k}{2}}$ & $11^2$ \\
					$\generado{1,i,\frac{1+k}{2},%
						\frac{i-j}{2}}$ & $11^2$
				\end{tabular}
			\end{center}
	\end{itemize}
\end{frame}

\begin{frame}{Clases de ideales}
	\begin{align*}
		\ideales{\cal O} & \,:=\,\big\{I\subset B\,:\,
			\Oder(I)=\cal O	\text{ , invertible}\big\}
		\text{ .}
	\end{align*}
	%
	Dados $I,J\in\ideales{\cal O}$, $I\sim J$, si existe $h\in B^\times$
	tal que $I=h\,J$.
	\begin{align*}
		\Class{\cal O} & \,:=\,B^\times\backslash\ideales{\cal O}
		\text{ .}
	\end{align*}
	%
	En general, no es un grupo. $H(\cal O):=\#\Class{\cal O}<\infty$.
	\begin{teoNorma}\label{thm:norma}
		Si $\cal O$ es de Eichler,
		\begin{math}
			B_+^\times :=\big\{h\in B^\times\,:\,\nrd(h)\gg 0\big\}
		\end{math},
		\begin{align*}
			\nrd & \,:\,B^\times_+\backslash\ideales{\cal O}
				\,\twoheadrightarrow\,\pClass{F}
			\text{ .}
		\end{align*}
		%
		Si $B$ es indefinida, es una biyecci\'{o}n (aproximaci\'{o}n
		fuerte). Si $B$ es definida,
		$\nrd:\,\Class{\cal O}\twoheadrightarrow\pClass F$.
	\end{teoNorma}
\end{frame}

\subsection{Formas cuaterni\'{o}nicas}\label{subsec:cuaternionicas}

\begin{frame}{Formas cuaterni\'{o}nicas: $B$ indefinida}
	Sean $B/F$ un \'{a}lgebra indefinida ($r=1$) y
	$v_1,\,\dots,\,v_n\in\lugares[\infty]F$:
	\begin{align*}
		B_{v_1}\,\simeq\,\MM[2\times 2](\bb R) & \quad\text{y}\quad
			B_{v_j}\,\simeq\,\bb H
			\quad\text{(\phantom)}j>1\text{\phantom().}
	\end{align*}
	%
	El grupo $B^\times_+=\big\{\nrd\gg 0\big\}$ act\'{u}a en $\hP$ v\'{\i}a
	$\gamma\mapsto\gamma_\infty\in\GLtp[2](\bb R)$. Dadas
	$f:\,\hP\rightarrow\bb C$,
	\begin{math}
		\gamma_\infty=
			\left[\begin{smallmatrix}
				a & b \\ c & d
			\end{smallmatrix}\right]
	\end{math},
	\begin{align*}
		f\operadormatrices{\null}{\gamma}(z) & \,=\,
			\frac{\det(\gamma_\infty)}{(cz+d)^2}\,
				f(\gamma_\infty\cdot z)
		\text{ .}
	\end{align*}
	%
\end{frame}

\begin{frame}{Formas cuaterni\'{o}nicas: $B$ indefinida (cont.)}
	Dado un orden de Eichler $\cal O\subset B$ de nivel $\frak N$,
	$\Gamma:=\cal O_+^\times$. El cociente
	\begin{align*}
		X^B_0(\frak N) & \,:=\,\Gamma\backslash\hP
	\end{align*}
	%
	es una curva compleja compacta (no hay c\'{u}spides) y conexa.

	\begin{defCuaternionicas}\label{def:cuaternionicasindef}
		$f:\,\hP\rightarrow\bb C$ es una \emph{forma cuaterni\'{o}%
		nica de peso $\peso 2$ para $\cal O$}, si
		\begin{enumerate}
			\item $f$ es holomorfa en $\hP$;
			\item para $\gamma\in\Gamma$,
				$f\operadormatrices{}{\gamma}=f$.
		\end{enumerate}
		%
		Denotamos este espacio por $\spitzH[B]2{\frak N}$.
	\end{defCuaternionicas}
\end{frame}

\begin{frame}{Operadores de Hecke}
	El espacio $\spitzH[B]2{\frak N}$ posee un producto interno y, para
	cada primo $\frak p\nmid\disc[B]\cdot\frak N$, operadores
	$T_{\frak p}:\,\spitzH[B]2{\frak N}\rightarrow\spitzH[B]2{\frak N}$
	tales que
	\begin{itemize}
		\item $T_{\frak p}T_{\frak q}=T_{\frak q}T_{\frak p}$ y
		\item
			\begin{math}
				\langle T_{\frak p}f,g\rangle=
					\langle f,T_{\frak p}g\rangle
			\end{math}.
	\end{itemize}
	%
	Si $\frak p=\generado p$ con $p\gg 0$, entonces
	\begin{align*}
		T_{\frak p}f & \,:=\,\sum_{i\in \oka F/\frak p}\,
			f\operadormatrices{}{\pi_i} \,+\,
			f\operadormatrices{}{\pi_\infty}
		\text{ ,}
	\end{align*}
	%
	donde $\pi_i,\pi_\infty\in\cal O$, $\nrd(\pi_i)=\nrd(\pi_\infty)=p$ y
	forman un sistema de representantes de
	\begin{align*}
		\Theta(\frak p) & \,:=\,\Gamma\backslash\Big\{
			\pi\in\cal O_+\,:\,\generado{\nrd(\pi)}=\frak p\Big\}
		\text{ .}
	\end{align*}
	%
\end{frame}

\begin{frame}{Formas cuaterni\'{o}nicas: $B$ definida}
	Sea $B/F$ un \'{a}lgebra definida ($r=0$). $B_{v_i}\simeq\bb H$, si
	$v_i\in\lugares[\infty]{F}$ y no hay acci\'{o}n en $\hP$.
	Sea $\cal O\subset B$ un orden de Eichler de nivel $\frak N$.

	\begin{defCuaternionicas}\label{def:cuaternionicasdef}
		Una \emph{forma modular cuaterni\'{o}nica de peso $\peso 2$ %
		para $\cal O$} es una funci\'{o}n
		$f:\,\ideales{\cal O}\rightarrow\bb C$ tal que $f(b\,I)=f(I)$
		para todo $b\in B^\times$. Denotamos este espacio
		$\modularH[B]2{\frak N}$.
	\end{defCuaternionicas}

	\begin{obsCuaternionicas}\label{obs:cuaternionicasdef}
		Sean $I\in\ideales{\cal O}$, $[I]$ la funci\'{o}n
		carcter\'{\i}stica de la clase.
		\begin{itemize}
			\item $[I]\in\modularH[B]2{\frak N}$;
			\item $\{[I_1],\,\dots,\,[I_H]\}$ es base de
				$\modularH[B]2{\frak N}$.
		\end{itemize}
		%
	\end{obsCuaternionicas}
\end{frame}

\begin{frame}{Operadores de Hecke}
	Dados $\frak p\nmid\disc[B]\cdot\frak N$ e $I\in\ideales{\cal O}$,
	\begin{align*}
		\cal T_{\frak p}(I) & \,:=\,
			\Big\{J\in\ideales{\cal O}\,:\,J\subset I,\,
				\nrd(J)=\frak p\,\nrd(I)\Big\}
			\text{ ,} \\
		T_{\frak p}[I] & \,:=\,\sum_{J\in\cal T_{\frak p}(I)}\,
			[J]
		\text{ .}
	\end{align*}
	%
	\begin{itemize}
		\item El espacio $\modularH2{\frak N}$ admite un producto
			interno respecto del cual los $T_{\frak p}$ son
			autoadjuntos.
		\item Existe $e_0\in\modularH2{\frak N}$, autofunci\'{o}n
			simult\'{a}nea para los $T_{\frak p}$, con autovalor
			$\idnorm\frak p+1$.
		\item
			\begin{math}
				\spitzH[B]2{\frak N}:=
					\Big\{f\in\modularH[B]2{\frak N}\,:\,
					\langle f,e_0\rangle=0\Big\}
			\end{math}.
	\end{itemize}
\end{frame}

\begin{frame}{La correspondencia de Jacquet-Langlands}
	\begin{teoCorrespondenciaJL}\label{thm:correspondenciajl}
		Sea $B/F$ un \'{a}lgebra de cuaterniones de discriminante
		$\disc[B]$ y sea $\frak N'\subset\oka F$ un ideal coprimo con
		$\disc[B]$. Entonces existe un morfismo inyectivo de
		m\'{o}dulos de Hecke
		\begin{align*}
			& \spitzH[B]2{\frak N'} \,\hookrightarrow\,
				\spitzH2{\disc[B]\cdot\frak N'}
		\end{align*}
		%
		cuya imagen consiste en las formas $f$ nuevas en los primos
		$\frak p\mid\disc[B]$.
	\end{teoCorrespondenciaJL}
\end{frame}

\begin{frame}{La correspondencia de Jacquet-Langlands (cont.)}
	Sean $\frak N=\frak p\frak q$ ($\frak p\not =\frak q$), $B/F$ con
	$\Ram(B)\cap\lugares[f]{F}=\{\frak p\}$. Por J-L,
	\begin{align*}
		\spitzH2{\frak p\frak q} & \,=\,
			\spitzH2{\frak p\frak q}^{\frak p-\neue}\,\oplus\,
			\spitzH2{\frak p\frak q}^{\frak p-\oude} \\
		& \,=\,\spitzH[B]2{\frak q}\,\oplus\,
			\Big(\iota_1\big(\spitzH2{\frak q}\big)\,+\,
				\iota_{\frak p}\big(\spitzH2{\frak q}\big)
				\Big)
		\text{ .}
	\end{align*}
	%
	Si $n=[F:\bb Q]=1\text{ o }2$, hay una \'{u}nica posibilidad.
	Si $n>2$ hay muchas \'{a}lgebras (ramificaci\'{o}n en $\infty$).

	Si, en cambio, $\frak N=\frak p^2$, no tenemos tantas opciones: debe
	ser $\disc[B]=1$ y $\Ram(B)\subset\lugares[\infty]F$. Si $n=1$ es
	imposible; si $n=2$, hay una \'{u}nica elecci\'{o}n.
\end{frame}

\begin{frame}{La correspondencia de Jacquet-Langlands (cont.)}
	\begin{obsCorrespondenciaJL}\label{obs:correspondenciajl}
		\begin{itemize}
			\item Si $B/F$ es indefinida,
				$f\in\spitzH[B]2{\frak N}$ tiene asociada una
				$r$-forma diferencial holomorfa en la
				\emph{variedad}
				$\cal O^\times_+\backslash\hP^r$;
			\item si $B$ es totalmente definida, $f$ es una
				funci\'{o}n en un \emph{espacio finito}.
		\end{itemize}
		%
	\end{obsCorrespondenciaJL}
	Hacer la elecci\'{o}n m\'{a}s sencilla y eficiente posible,
	$r=0\text{ o }1$:
	\begin{itemize}
		\item si $2\mid n=[F:\bb Q]$, tomar
			$\Ram(B)=\lugares[\infty]F$;
		\item si $2\nmid n$, tomar
			$\Ram(B)=\lugares[\infty]F\setmin\{v_1\}$ ($n>2$).
	\end{itemize}
	%
	En estos casos, $\spitzH2{\frak N}\simeq\spitzH[B]2{\frak N}$.
\end{frame}


\section{Ejemplos}\label{sec:ejemplos}
\theoremstyle{remark}

%-------------

\begin{frame}{M\'{e}todo definido}
	$F=\bb Q(\sqrt 5)$, $B=\quatalg[F]{-1,-1}$,
	$\Ram(B)=\lugares[\infty]{F}$.
	\begin{itemize}
		\item Si $\cal O_0(1)$ es maximal, $H(\cal O_0(1))=1$ y
			$\dim{\spitzH[B]2{1}}=0$ (sobre $\bb Q$,
			$\spitz4{1}=0$);
		\item sobre $F$, $11=\frak n_1\frak n_2$ y
			\begin{math}
				\spitzH[B]2{\frak n_1}=0=
					\spitzH[B]2{\frak n_2}
			\end{math}, tambi\'{e}n;
		\item pero $H(\cal O_0(\frak n_1^2))=3$ y
			\begin{math}
				\spitzH[B]2{\frak n_1^2}=
					\spitzH[B]2{\frak n_1^2}^\neue\not =0
			\end{math}.
	\end{itemize}
\end{frame}

\begin{frame}{M\'{e}todo definido}
	\begin{align*}
		\dim{\spitzH[B]2{\frak n_1^2}} & \,=\,2
	\end{align*}
	%
	\begin{center}
		\begin{tabular}{c|ccccccccc}%cc}
			$\idnorm\,\frak p$ & $4$ & $5$ & $9$ &
				% $11$ & $11$ &
				$19$ & $19$ & $29$ & $29$ & $31$ & $31$ \\
			\hline
			$a_{\frak p}(f)$ & $t$ & $-t$ & $-1$ &
				% $-2t$ & $0$ &
				$4t$ & $-2$ & $-3$ & $-5t$ & $2t$ & $2t$
		\end{tabular}
	\end{center}
	($t^2-3=0$).
	\begin{align*}
		T_{\frak p_4} \,=\,
			\begin{bmatrix}
				1 & -1 \\ -2 & -1
			\end{bmatrix} & \quad\text{,}\quad
		T_{\frak p_5} \,=\,
			\begin{bmatrix}
				-1 & 1 \\ 2 & 1
			\end{bmatrix} \quad\text{,}\quad
		T_{\frak p_9} \,=\,
			\begin{bmatrix}
				-1 & 0 \\ 0 & -1
			\end{bmatrix}
	\end{align*}
	%
\end{frame}

\begin{frame}{M\'{e}todo definido}
	\begin{align*}
		\dim{\spitzH[B]2{11}} & \,=\,\dim{\spitzH[B]2{11}^\neue}\,=\,3
	\end{align*}
	%
	\begin{center}
		\begin{tabular}{c|ccccccc}%cc}%cc}
			$\idnorm\,\frak p$ & $4$ & $5$ & $9$ &
				% $11$ & $11$ &
				$19$ & $19$ & $29$ & $29$ \\
				% & $31$ & $31$ \\
			\hline
			$a_{\frak p}(f)$ & $0$ & $1$ & $-5$ &
				% $1$ & $1$ &
				$0$ & $0$ & $0$ & $0$ \\
				% & $7$ & $7$ \\
			$a_{\frak p}(g)$ & $t$ & $-t-1$ & $-t+3$ &
				% $-1$ & $-1$ &
				$-4$ & $-4$ & $2t-4$ & $2t-4$
				% & $-t+1$ & $-t+1$
		\end{tabular}
	\end{center}
	($t^2-t-8=0$).\hfill
	% \begin{flushright}
		\begin{tabular}{cc}
			$31$ & $31$ \\
			\hline
			$7$ & $7$ \\
			$-t+1$ & $-t+1$
		\end{tabular}
	% \end{flushright}
	\begin{align*}
		T_{\frak p_4} \,=\,
			\begin{bmatrix}
				0 & 0 & 0 \\ 0 & 4 & -4 \\ 5 & 1 & -3
			\end{bmatrix} & \quad\text{,}\quad
		T_{\frak p_5} \,=\,
			\begin{bmatrix}
				1 & 0 & 0 \\ 5 & -5 & 4 \\ 0 & -1 & 2
			\end{bmatrix}
	\end{align*}
	%
\end{frame}

\begin{frame}{M\'{e}todo indefinido}
	$F=\bb Q(w)$, % donde $w^3 - w^2 - 8\,w + 7=0$.
	$B=\quatalg[F]{5-w^2,1-w^2}$ ramifica en dos de los tres lugares
	reales. Sea $\cal O$ un orden de Eichler de nivel $\frak n$.
	\begin{itemize}
		\item Si $\pClass{F}=\{1,\frak a\}$, existe
			$J_{\frak a}\in\ideales{\cal O}$ con
			$\nrd(J_{\frak a})=\frak a$.
		\item $\cal O_1=\cal O$, $\cal O_{\frak a}=\Oizq(J_{\frak a})$.
		\item $\Gamma_1=\inc[\infty](\cal O_{1,+}^\times)$,
			\begin{math}
				\Gamma_{\frak a}=
					\inc[\infty](\cal O_{\frak a,+}^\times)
			\end{math} act\'{u}an en $\hP$.
	\end{itemize}
	\begin{align*}
		X_0^B(\frak n) & \,=\,\Gamma_1\backslash\hP
			\,\sqcup\,\Gamma_{\frak a}\backslash\hP
		\text{ .}
	\end{align*}
	%
	Por Eichler-Shimura,
	\begin{align*}
		\spitzH[B]2{\frak n}\,\oplus\,\lconj{\spitzH[B]2{\frak n}}
			& \,=\,\mathsf H^1\big(X_0^B(\frak n),\bb C\big)
			\,=\,\mathsf H^1\big(X(\Gamma_1),\bb C\big)\,\oplus\,
				\mathsf H^1\big(X(\Gamma_{\frak a}),\bb C\big)
		\text{ .}
	\end{align*}
	%
\end{frame}

\begin{frame}{M\'{e}todo indefinido}
	Sobre $F$, $31=\frak n_1\frak n_2$, con $\idnorm(\frak n_1)=31$,
	$\idnorm(\frak n_2)=31^2$.
	\begin{align*}
		\dim{\spitzH[B]2{\frak n_1}} \,=\, 86 & \quad\text{,}\quad
		\dim{\spitzH[B]2{\frak n_2}} \,=\, 2722 \text{ ,} \\
		\dim{\spitzH[B]2{31}^\neue} & \,=\, 81602 \text{ .}
	\end{align*}
	%
	$\spitzH[B]2{\frak n_1}$ se descompone como suma de subespacios
	Hecke-irreducibles de dimensiones $1$, $1$, $2$, $2$, $2$, $8$, $24$ y
	$46$.
	\begin{center}
		\begin{tabular}{c|ccccccccc}
			$\idnorm\,\frak p$ & $5$ & $7$ &
				$8$ & $11$ & $13$ & $17$ & $23$ & $23$ & $23$
					\\
			\hline
			$a_{\frak p}(f_1)$ & $-3$ & $4$ &
				$3$ & $0$ & $2$ & $-3$ & $5$ & $-8$ & $3$ \\
			$a_{\frak p}(f_2)$ & $-3$ & $-4$ &
				$3$ & $0$ & $-2$ & $3$ & $-5$ & $-8$ & $-3$
		\end{tabular}
	\end{center}
\end{frame}


\section*{}
\begin{frame}
	\begin{center}
		\Large{!`Muchas gracias!}
	\end{center}
\end{frame}

\end{document}
